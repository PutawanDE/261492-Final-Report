\maketitle
\makesignature

\ifproject
\begin{abstractTH}
เพชฌฆาตแม่มด: เกมแนวสยองขวัญแบบผู้เล่นหลายคน พัฒนาขึ้นเพื่อนำเสนอเกมสยองขวัญที่ให้ความบันเทิง มีแรงบันดาลใจจากความเชื่อยุคกลางที่มีอยู่ว่า ปีศาจปลอมแปลงเป็นสัตว์ได้หลายชนิด อาศัยอยู่ในป่ารกร้าง ผู้เล่นจะต้องต่อสู้กัน ทั้งฝ่ายปราบแม่มดและแม่มด โดยทีมปราบแม่มดต้องทำงานร่วมกันเพื่อทำภารกิจ ส่วนแม่มดมีหน้าที่ขัดขวางฝั่งตรงข้าม มีความสามารถที่จะแปลงกายเป็นสัตว์เพื่ออำพรางและขัดขวางภารกิจได้ เกมนี้พัฒนาโดยใช้ Unreal Engine เพื่อให้ผู้เล่นได้สัมผัสประสบการณ์ที่สมจริงผ่านมุมมองบุคคลที่หนึ่ง
\end{abstractTH}

\begin{abstract}
Witch Hunter: A Multiplayer Horror Game is developed to present an entaining horror game. It is inspired by medieval beliefs that demons can disguise themselves as animals and live in the wilderness. Players must fight each other, with one side playing as a team of demon hunters and the other as the witch. The demon hunters must work together to complete their mission, while the witch has the ability to transform into animals to obstruct and interfere with the mission. The game is developed using the Unreal Engine to provide players with an immersive first-person perspective.
\end{abstract}
\iffalse
\begin{dedication}
This document is dedicated to all Chiang Mai University students.

Dedication page is optional.
\end{dedication}
\fi % \iffalse

\begin{acknowledgments}
โครงงานเรื่อง เพชฌฆาตแม่มด: เกมแนวสยองขวัญแบบผู้เล่นหลายคน จะไม่สามารถสำเร็จลุล่วงลงได้ ถ้าไม่ได้รับความกรุณาจาก ผศ.ดร.กานต์ ปทานุคม
อาจารย์ที่ปรึกษาที่ได้เสียสละเวลาให้คำแนะนำ และข้อเสนอแนะที่มีคุณค่ามากมายตลอดการพัฒนาโครงงานนี้ รวมถึงคณะกรรมการสอบโครงงานทุกท่าน รศ.ดร.ปฏิเวธ วุฒิสารวัฒนา
และ รศ.ดร.ศักดิ์กษิต ระมิงค์วงศ์​ ที่ได้ให้คำติชมและข้อเสนอแนะที่สำคัญ

ขอขอบคุณภาควิชาวิศวกรรมคอมพิวเตอร์ มหาวิทยาลัยเชียงใหม่ ที่ได้สนับสนุนความรู้และทรัพยาการให้สามารถทำโครงงานนี้ได้สำเร็จ

ขอขอบคุณเพื่อน ๆ ที่ได้มีส่วนร่วมในการทดสอบเกม Witch Hunter ตั้งแต่ตอนที่เล่นแทบไม่ได้จนกระทั่งเสร็จสมบูรณ์ ขอขอบคุณที่ได้ทดลองเล่น ให้คำแนะนำ กำลังใจ และข้อเสนอแนะที่สำคัญ

\acksign{2024}{3}{26}
\end{acknowledgments}%
\fi % \ifproject

\contentspage

\ifproject
\figurelistpage

\tablelistpage
\fi % \ifproject

% \abbrlist % this page is optional

% \symlist % this page is optional

% \preface % this section is optional
