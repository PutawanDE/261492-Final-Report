\chapter{\ifenglish Background Knowledge and Theory\else ทฤษฎีที่เกี่ยวข้อง\fi}

การทำโครงงาน เริ่มต้นด้วยการศึกษาค้นคว้า ทฤษฎีที่เกี่ยวข้อง หรือ งานวิจัย/โครงงาน ที่เคยมีผู้นำเสนอไว้แล้ว ซึ่งเนื้อหาในบทนี้ก็จะเกี่ยวกับการอธิบายถึงสิ่งที่เกี่ยวข้องกับโครงงาน เพื่อให้ผู้อ่านเข้าใจเนื้อหาในบทถัดๆ ไปได้ง่ายขึ้น

\section{เกมสยองขวัญแบบผู้เล่นหลายคนในตลาด}
\subsection{Dead Space 3}

Dead Space 3 เป็นเกมสยองขวัญที่ให้ผู้เล่นสามารถเล่นโดยใช้ความร่วมมือจากผู้เล่น 2 คน โดยต่อสู้กับ Necromorph โดยรวมแล้วเป็นเกมสยองขวัญเอาชีวิตรอดที่ตึงเครียดและน่าตื่นเต้นที่นำเสนอประสบการณ์ความร่วมมือที่สมจริง เรื่องราวที่น่าดึงดูด และกลไกการเล่นเกมที่หลากหลายเพื่อให้ผู้เล่นมีส่วนร่วมและลุ้นตาม \cite{DeadSpace3}


\subsection{Left 4 Dead 2}

Left 4 Dead 2 เป็นเกมสยองขวัญเอาชีวิตรอดจากโลกที่เต็มไปด้วยฝูงผีดิบ โดยเล่นด้วยกันได้ตั้งแต่ 1 ถึง 4 คน ผู้เล่นจะรับบทบาทเป็นผู้เอาชีวิตรอด มีอาวุธให้ผู้เล่นได้เลือกตามความถนัดของตน โดยรวมแล้วเป็นเกมยิงแบบร่วมมือที่รวดเร็วและเข้มข้น ซึ่งนำเสนอการผสมผสานระหว่างกลยุทธ์ การทำงานเป็นทีม และความสยองขวัญ ลุ้นระทึกเอาชีวิตรอด กับรูปแบบการเล่นที่น่าดึงดูดและตัวละครที่น่าจดจำ \cite{L4D2}



\subsection{Phasmophobia}

Phasmophobia เกมนี้ให้ผู้เล่นสวมบทบาทเป็นนักล่าผี และ รองรับผู้เล่นสูงสุด 4 คนที่ทำงานร่วมกันเพื่อสำรวจสถานที่และ รวบรวมหลักฐานเหตุการณ์เหนือธรรมชาติ โดยรวมแล้วเป็นเกมสยองขวัญที่ไม่เหมือนใครและน่าดึงดูด ซึ่งรวมเอาองค์ประกอบของการสืบสวน การเอาชีวิตรอด และการเล่นเกมแบบร่วมมือกัน เป็นเกมที่ต้องใช้การสื่อสาร การทำงานเป็นทีม และการคิดเชิงกลยุทธ์เพื่อให้ประสบความสำเร็จ \cite{Phasmophobia}


\subsection{The Forest}

The Forest เกมสยองขวัญเอาชีวิตรอด open world โดยผู้เล่นทำงานร่วมกันเพื่อสร้างที่พักพิง รวบรวมทรัพยากร และ ป้องกันมนุษย์กลายพันธุ์ โดยรวมแล้วเป็นเกมสยองขวัญเอาชีวิตรอดที่น่าตื่นเต้นและท้าทายที่นำเสนอการผสมผสานที่ไม่เหมือนใครระหว่างการสำรวจ การสร้าง และการต่อสู้ ด้วยบรรยากาศที่ตึงเครียด ศัตรูที่ท้าทาย และโลกที่สมจริง \cite{TheForest}



\subsection{Devour}

Devour เป็นเกมสยองขวัญเอาชีวิตรอดแบบอาศัยความร่วมมือของผู้เล่น 1-4 คน แต่ละเกมยาวหนึ่งชั่วโมง หน้าที่ของผู้เล่น คือ ทำภารกิจในรูปแบบต่าง ๆ เพื่อเอาตัวรอดจากสมาชิกลัทธิที่ถูกผีสิง เช่น การเผาแพะเพื่อทำพิธีกรรม โดยผู้เล่นจะต้องรวบรวมทรัพยากรต่าง ๆ เพื่อใช้ในการทำภารกิจ และต้องระวังศัตรูที่อยู่รอบตัว \cite{Devour}

\section{เครื่องมือที่ใช้ในการพัฒนา}
\subsection{ Unreal Engine 5}
Unreal Engine 5 เป็นเครื่องมือพัฒนาเกมและซอฟต์แวร์สำหรับสร้างสภาพแวดล้อมแบบสามมิติ (3D) ที่ถูกพัฒนาโดย Epic Games ซึ่งเป็นบริษัทผู้พัฒนาเกมระดับโลกและเจ้าของหรือผู้จัดการสิทธิบัตรของหลายแพลตฟอร์มเกมที่สำคัญ เช่น Fortnite และ Gears of War Unreal Engine 5 เป็นเวอร์ชันล่าสุดของเครื่องมือการพัฒนาเกมของค่ายนี้และถูกประกาศเปิดตัวครั้งแรกในปี 2020

\subsection{Blender}
Blender เป็นโปรแกรมซอฟต์แวร์สำหรับสร้างกราฟิก 3 มิติ (3D) ที่มีความสามารถหลากหลาย เช่น การออกแบบและสร้างโมเดล 3 มิติ, การทำแอนิเมชัน, การสร้างภาพนิ่ง, และการสร้างภาพเคลื่อนไหว (วิดีโอ) ซึ่งมีความยืดหยุ่นและมีฟังก์ชันมากมายที่ช่วยให้นักออกแบบสามารถสร้างงานศิลปะที่มีคุณภาพสูงได้ โดย \\Blender เป็นโปรแกรมฟรีและเปิดตัวสำหรับใช้งาน มันสามารถใช้ได้กับหลายแพลตฟอร์ม เช่น Windows, macOS, และ Linux และมีชุดเครื่องมือที่ครอบคลุมตั้งแต่การสร้างโมเดลจนถึงการนำเสนองานออกมาในรูปแบบที่ต้องการได้ โดยสามารถนำผลงานที่สร้างขึ้นใน Blender ไปใช้ในหลายอุตสาหกรรม เช่น ภาพยนตร์, การทำเกม, การออกแบบผลิตภัณฑ์, และการพิมพ์ 3 มิติ

\subsection{GitLFS}
GitLFS ย่อมาจาก Git Large File Storage ซึ่งเป็นระบบการจัดการไฟล์ขนาดใหญ่ใน Git โดยเฉพาะ ซึ่ง Git นั้นเป็นระบบควบคุมเวอร์ชัน (Version Control System) ที่ใช้สำหรับการจัดการโค้ด ซึ่งมักจะใช้ในการพัฒนาซอฟต์แวร์ แต่มันไม่ได้เหมาะสำหรับการจัดเก็บไฟล์ขนาดใหญ่ เช่น ไฟล์เสียงวิดีโอหรือไฟล์ภาพที่มีขนาดใหญ่มาก
GitLFS ถูกพัฒนาขึ้นเพื่อแก้ไขปัญหานี้ โดย GitLFS ช่วยในการจัดเก็บไฟล์ขนาดใหญ่ในเครื่องเซิร์ฟเวอร์อย่างมีประสิทธิภาพ โดยไม่ทำให้เครื่องเซิร์ฟเวอร์โหลดไฟล์ทั้งหมดเมื่อมีการดึงโค้ดที่มีไฟล์ขนาดใหญ่จาก Git repository แทนที่จะเก็บไฟล์โดยตรงใน repository เช่นเดียวกับไฟล์ทั่วไป GitLFS จะเก็บไฟล์ขนาดใหญ่ในพื้นที่ที่เรียกว่า LFS storage ซึ่งส่วนใหญ่จะอยู่ในเครื่องเซิร์ฟเวอร์ภายนอก
การใช้ GitLFS ช่วยลดขนาดของ Git repository ช่วยในการจัดการไฟล์ขนาดใหญ่ที่มีการเปลี่ยนแปลงบ่อย ๆ และทำให้การทำงานร่วมกันในโปรเจกต์ที่มีไฟล์ขนาดใหญ่เป็นไปอย่างมีประสิทธิภาพมากขึ้น

\subsection{GitLab}
GitLab เป็นแพลตฟอร์มการพัฒนาซอฟต์แวร์ที่ให้บริการเครื่องมือต่างๆ ที่เกี่ยวข้องกับการจัดการโครงการพัฒนาซอฟต์แวร์ ซึ่งรวมถึงระบบควบคุมเวอร์ชัน Git ระบบติดตามปัญหา (Issue Tracking System) ระบบวิกิ (Wiki System) และระบบการดำเนินการต่าง ๆ ที่เกี่ยวข้องกับการพัฒนาซอฟต์แวร์และการทำงานในโครงการซอฟต์แวร์
นอกจากนี้ GitLab ยังเป็นเครื่องมือที่ให้บริการพื้นที่เก็บโค้ด (Repository Hosting Service) และให้บริการต่าง ๆ เกี่ยวกับการจัดการ Source Code เช่น การสร้างและจัดการซอร์สโค้ด การจัดการเวอร์ชันของโค้ด การเปิดเผยโค้ดต่าง ๆ Collaborative Coding การตรวจสอบโค้ด (Code Review) และการจัดการเครื่องมือต่าง ๆ ที่เกี่ยวข้องกับการพัฒนาซอฟต์แวร์ ทำให้ GitLab เป็นอีกทางเลือกหนึ่งสำหรับทีมที่ต้องการจัดการโครงการพัฒนาซอฟต์แวร์ของตน
นอกจาก GitLab Community Edition ที่เป็นเวอร์ชันโอเพ่นซอร์ส (Open Source) แล้ว ยังมี GitLab Enterprise Edition ที่เป็นเวอร์ชันเสียเงินที่มีฟีเจอร์เพิ่มเติมที่เหมาะสำหรับองค์กรหรือธุรกิจที่ต้องการระบบควบคุมเวอร์ชันของตนเองและความมั่นคงของการให้บริการ

\subsection{Mixamo}
Mixamo เป็นบริการออนไลน์ที่เชื่อมโยงกับ Adobe Creative Cloud ที่มุ่งเน้นการสร้างและปรับแต่งตัวละครแบบ 3 มิติ (3D characters) และการทำแอนิเมชัน (animation) อย่างรวดเร็วและง่ายดาย ในช่วงเวลาที่ผ่านมา Mixamo เป็นที่นิยมในวงการเกมและอินเตอร์เน็ตเพราะการใช้งานที่สะดวกและประหยัดเวลา โดยเฉพาะสำหรับนักพัฒนาเกม ซึ่งสามารถนำตัวละครและแอนิเมชันจาก Mixamo ไปใช้ในโปรเจกต์เกมของตนได้.

\subsection{Character Creator}
Character Creator เป็นซอฟต์แวร์ที่ใช้สร้างตัวละครสามมิติ (3D characters) ที่มีความสมจริงสำหรับใช้ในเกม ภาพนิ่ง หรือวิดีโออนิเมชัน ซอฟต์แวร์นี้ถูกพัฒนาโดย Reallusion และมีความสามารถที่ให้ผู้ใช้สร้างตัวละครที่หลากหลายในแง่มุมต่า งๆ ซึ่งรวมถึงรายละเอียดที่มากมาย เช่น รูปร่างของร่างกาย สีผิว ทรงผม ส่วนสวมใส่ และอื่น ๆ ที่เกี่ยวข้องกับลักษณะภายนอกของตัวละคร
Character Creator มีเครื่องมือที่ทำให้การสร้างตัวละครในระดับที่มีความสมจริงสูงมากขึ้น โดยสามารถปรับแต่งรายละเอียดต่างๆ ของตัวละครได้อย่างอิสระ เช่น การปรับแต่งรูปร่าง, สีผิว, ทรงผม, สร้างเสื้อผ้าและอุปกรณ์ต่างๆ เพื่อให้ตัวละครมีความเป็นเอกลักษณ์และสอดคล้องกับความต้องการของโครงการ
นอกจากนี้ Character Creator ยังมีการใช้งานร่วมกับโปรแกรมอื่นๆ เช่น Maya, Blender, Unity, และ Unreal Engine เพื่อให้ผู้ใช้สามารถนำเข้าตัวละครที่สร้างขึ้นมาใช้ในโปรเจกต์พัฒนาซอฟต์แวร์ต่าง ๆ ได้อย่างสะดวกสบาย ทำให้เป็นเครื่องมือที่มีประสิทธิภาพและความสามารถในการสร้างตัวละครสามมิติที่มีคุณภาพสูง


\section{\ifenglish%
\ifcpe CPE \else ISNE \fi knowledge used, applied, or integrated in this project
\else%
ความรู้ตามหลักสูตรซึ่งถูกนำมาใช้หรือบูรณาการในโครงงาน
\fi
}

\begin{enumerate}
  \item Software Engineering: ใช้ความรู้เรื่องการจัดการการพัฒนาซอฟต์แวร์
  \item Computer Networking: ใช้ความรู้ในเรื่องเครือข่ายคอมพิวเตอร์ในการพัฒนาส่วนการเล่นหลายคน
  \item Object-oriented Programming: ใช้ความรู้ในการเขียนโปรแกรมเชิงวัตถุในการเขียนโปรแกรม
\end{enumerate}


\section{\ifenglish%
Extracurricular knowledge used, applied, or integrated in this project
\else%
ความรู้นอกหลักสูตรซึ่งถูกนำมาใช้หรือบูรณาการในโครงงาน
\fi
}


\begin{enumerate}
  \item การพัฒนาเกมด้วย Unreal Engine โดยใช้ Blueprint และภาษา C++
  \item การออกแบบวิธีการเล่นและฉาก
  \item การออกแบบงานศิลป์ในเกม
  \item การสร้างเนื้อเรื่องและเนื้อหาในเกม
\end{enumerate}

